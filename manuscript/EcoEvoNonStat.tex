% Options for packages loaded elsewhere
\PassOptionsToPackage{unicode}{hyperref}
\PassOptionsToPackage{hyphens}{url}
%
\documentclass[
]{article}
\usepackage{amsmath,amssymb}
\usepackage{lmodern}
\usepackage{iftex}
\ifPDFTeX
  \usepackage[T1]{fontenc}
  \usepackage[utf8]{inputenc}
  \usepackage{textcomp} % provide euro and other symbols
\else % if luatex or xetex
  \usepackage{unicode-math}
  \defaultfontfeatures{Scale=MatchLowercase}
  \defaultfontfeatures[\rmfamily]{Ligatures=TeX,Scale=1}
\fi
% Use upquote if available, for straight quotes in verbatim environments
\IfFileExists{upquote.sty}{\usepackage{upquote}}{}
\IfFileExists{microtype.sty}{% use microtype if available
  \usepackage[]{microtype}
  \UseMicrotypeSet[protrusion]{basicmath} % disable protrusion for tt fonts
}{}
\makeatletter
\@ifundefined{KOMAClassName}{% if non-KOMA class
  \IfFileExists{parskip.sty}{%
    \usepackage{parskip}
  }{% else
    \setlength{\parindent}{0pt}
    \setlength{\parskip}{6pt plus 2pt minus 1pt}}
}{% if KOMA class
  \KOMAoptions{parskip=half}}
\makeatother
\usepackage{xcolor}
\IfFileExists{xurl.sty}{\usepackage{xurl}}{} % add URL line breaks if available
\IfFileExists{bookmark.sty}{\usepackage{bookmark}}{\usepackage{hyperref}}
\hypersetup{
  pdftitle={Linking evolutionary and ecological theory illuminates non-equilibrium biodiversity},
  hidelinks,
  pdfcreator={LaTeX via pandoc}}
\urlstyle{same} % disable monospaced font for URLs
\usepackage[margin=1in]{geometry}
\usepackage{graphicx}
\makeatletter
\def\maxwidth{\ifdim\Gin@nat@width>\linewidth\linewidth\else\Gin@nat@width\fi}
\def\maxheight{\ifdim\Gin@nat@height>\textheight\textheight\else\Gin@nat@height\fi}
\makeatother
% Scale images if necessary, so that they will not overflow the page
% margins by default, and it is still possible to overwrite the defaults
% using explicit options in \includegraphics[width, height, ...]{}
\setkeys{Gin}{width=\maxwidth,height=\maxheight,keepaspectratio}
% Set default figure placement to htbp
\makeatletter
\def\fps@figure{htbp}
\makeatother
\setlength{\emergencystretch}{3em} % prevent overfull lines
\providecommand{\tightlist}{%
  \setlength{\itemsep}{0pt}\setlength{\parskip}{0pt}}
\setcounter{secnumdepth}{-\maxdimen} % remove section numbering
\newlength{\cslhangindent}
\setlength{\cslhangindent}{1.5em}
\newlength{\csllabelwidth}
\setlength{\csllabelwidth}{3em}
\newlength{\cslentryspacingunit} % times entry-spacing
\setlength{\cslentryspacingunit}{\parskip}
\newenvironment{CSLReferences}[2] % #1 hanging-ident, #2 entry spacing
 {% don't indent paragraphs
  \setlength{\parindent}{0pt}
  % turn on hanging indent if param 1 is 1
  \ifodd #1
  \let\oldpar\par
  \def\par{\hangindent=\cslhangindent\oldpar}
  \fi
  % set entry spacing
  \setlength{\parskip}{#2\cslentryspacingunit}
 }%
 {}
\usepackage{calc}
\newcommand{\CSLBlock}[1]{#1\hfill\break}
\newcommand{\CSLLeftMargin}[1]{\parbox[t]{\csllabelwidth}{#1}}
\newcommand{\CSLRightInline}[1]{\parbox[t]{\linewidth - \csllabelwidth}{#1}\break}
\newcommand{\CSLIndent}[1]{\hspace{\cslhangindent}#1}
\usepackage{xr}
\externaldocument{EcoEvoNonStat_supp}
\usepackage{lineno}
\ifLuaTeX
  \usepackage{selnolig}  % disable illegal ligatures
\fi

\title{Linking evolutionary and ecological theory illuminates
non-equilibrium biodiversity}
\author{}
\date{\vspace{-2.5em}}

\begin{document}
\maketitle

\linenumbers

\begin{quote}
\hypertarget{abstract}{%
\subsubsection{Abstract}\label{abstract}}
\end{quote}

\begin{quote}
Whether or not biodiversity dynamics tend toward stable equilibria
remains an unsolved question in ecology and evolution with important
implications for our understanding of diversity and its conservation.
Phylo/population genetic models and macroecological theory represent two
primary lenses through which we view biodiversity. While
phylo/population genetics provide an averaged view of changes in
demography and diversity over timescales of generations to geological
epochs, macroecology provides an ahistorical description of commonness
and rarity across contemporary co-occurring species. Our goal is
leverage advances in community-wide high throughput sequencing
technology, specifically metabarcoding, to combine evolutionary and
macroecological approaches to gain novel insights into the
non-equilibrium nature of biodiversity. We help guide near future
research with a call for bioinformatic advances and an outline of
quantitative predictions made possible by our approach. \newline
\newline
\end{quote}

\hypertarget{non-equilibrium-inference-and-theory-in-ecology-and-evolution}{%
\section{Non-equilibrium, inference, and theory in ecology and
evolution}\label{non-equilibrium-inference-and-theory-in-ecology-and-evolution}}

The idea of an ecological and evolutionary equilibrium has pervaded
studies of biodiversity from geological to ecological, and from global
to local (Sepkoski 1984; Chesson 2000; Hubbell 2001; Tilman 2004;
Rabosky 2009; Harte 2011). The consequences of non-equilibrium dynamics
for biodiversity are not well understood and the need to understand them
is critical with anthropogenic pressures forcing biodiversity into
states of rapid transition (Blonder \emph{et al.} 2015). Non-equilibrial
processes could profoundly inform conservation in ways only just
beginning to be explored (Wallington \emph{et al.} 2005).

Biodiversity theories based on assumptions of equilibrium, both
mechanistic (Chesson 2000; Hubbell 2001; Tilman 2004) and statistical
\textbf{(see the Glossary)} (Pueyo \emph{et al.} 2007; Harte 2011) have
found success in predicting ahistorical patterns of diversity such as
the species abundance distribution (Hubbell 2001; Harte 2011; White
\emph{et al.} 2012) and the species area relationship (Hubbell 2001;
Harte 2011). These theories assume a macroscopic equilibrium in terms of
these coarse-grained metrics, as opposed to focusing on details of
species identity (such as in Blonder \emph{et al.} (2015)), although
macroscopic and microscopic approaches are not mutually exclusive.
Nonetheless, the equilibrium assumed by these theories is not realistic
(Ricklefs 2006), and many processes, equilibrial or otherwise, can
generate the same macroscopic, ahistorical predictions (McGill \emph{et
al.} 2007).

Tests of equilibrial ecological theory alone will not allow us to
identify systems out of equilibrium, nor permit us to pinpoint the
mechanistic causes of any non-equilibrial processes. The dynamic natures
of evolutionary innovation and landscape change suggest that ecological
theory could be greatly enriched by synthesizing its insights with
inference from population genetic theory that explicitly accounts for
history. This would remedy two shortfalls of equilibrial theory: 1) if
theory fits observed ahistorical patterns but the implicit dynamical
assumptions were wrong, we would make the wrong conclusion about the
equilibrium of the system; 2) if theories do not fit the data we cannot
know why unless we have a perspective on the temporal dynamics
underlying those data.

No efforts to date have tackled these challenges. We propose that
combining insights from ecological theory and inference of evolutionary
and demographic change from genetic data will allow us to understand and
predict the consequences of non-equilibrial processes in governing the
current and future states of ecological assemblages. The advent of high
throughput sequencing-enabled metabarcoding has made unprecedented data
available about the biodiversity of lineages from microbes to arthropods
(Taberlet \emph{et al.} 2012; Ji \emph{et al.} 2013; Zhou \emph{et al.}
2013; Bohmann \emph{et al.} 2014; Gibson \emph{et al.} 2014; Dodsworth
2015; Leray \& Knowlton 2015; Linard \emph{et al.} 2015; Shokralla
\emph{et al.} 2015; Venkataraman \emph{et al.} 2015; Liu \emph{et al.}
2016). These metabarcoding data have great potential to yield synthetic
insight across ecology and evolution. However, to draw such insights we
need a new tool set of bioinformatic methods \textbf{(Box box:dry)} and
meaningful predictions \textbf{(section ref sec:pred )} grounded in
theory to make use of those data. We present the foundation of this
methedological tool set here.

\hypertarget{ecological-theories-and-non-equilibrium}{%
\section{Ecological theories and
non-equilibrium}\label{ecological-theories-and-non-equilibrium}}

Neutral and statistical theories in ecology focus on macroscopic
patterns, and equilibrium is presumed to be relevant to those patterns.
Our goal throughout is not to validate neutral or statistical
theories---quite the opposite, we propose new data dimensions, namely
genetics, to help better test alternative hypotheses against these null
theories, thereby gaining insight into what non-neutral and
non-statistical mechanisms are at play in systems of interest.

Non-neutral and non-statistical models (e.g., Tilman (2004); Chesson
(2000)) also invoke ideas of equilibrium in their derivation. However,
these equilibria focus on the micro-scale details of species
interactions, which could in fact lead to instability and
non-equilibrium at larger scales (cite: romGEB, critical transition in
food web stuff from Beth, random matrix theory). Thus, detailed species
interactions could in fact be drivers of non-equilibrium and thus
interesting hypotheses to test as alternatives to neutral/statistical
models.

To use neutral/statistical theories as null models, we need a robust
measure of goodness of fit. The emerging consensus is that
likelihood-based test statistics should be preferred (Baldridge \emph{et
al.} 2016). The ``exact test'' of Etienne (2007) has been extended by
Rominger \& Merow (2017) into a simple z-score which can parsimoniously
describe the goodness of fit between theory and pattern. We advocate its
use in our proposed framework.

The unified neutral theory of biodiversity (UNTB) (Hubbell 2001) is a
useful null because it assumes that one mechanism---demographic
drift---drives community assembly. Equilibrium occurs when homogeneous
stochastic processes of birth, death, speciation and immigration have
reached stationarity. Thus neutrality in ecology is analogous to neutral
drift in population genetics (Hubbell 2001) (also cite molecular neutral
theory).

Rather than assuming any one mechanism dominates the assembly of
populations into a community, statistical theories assume all mechanisms
could be valid, but their unique influences have been lost to the
enormity of the system and thus the outcome of assembly is a community
in statistical equilibrium {[}Harte (2011); pueyo2007{]}. The
mechanistic agnosticism is what makes statistical theories useful nulls.
These statistical theories are also consistent with niche-based
equilibria (Pueyo \emph{et al.} 2007; Neill \emph{et al.} 2009) if the
complicated, individual or population level models with many mechanistic
drivers were to be upscaled to entire communities.

The maximum entropy theory of ecology (METE) (Harte 2011) derives its
predictions by condensing the many bits of mechanistic information down
into ecological state variables and then mathematically maximizing
information entropy conditional on those state variables. METE can
predict multiple ahistorical patterns, including distributions of
species abundance, body size, spatial aggregation, and trophic links
(Harte 2011; Rominger \emph{et al.} 2015; Rominger \& Merow 2017),
making for a stronger null theory (McGill 2003). However, multiple
dynamics can still map to this handful of metrics (McGill \emph{et al.}
2007) and while extensive testing often supports METE's predictions
{[}Harte (2011); white2012; xiao2015{]} at single snapshots in time,
METE fails to match observed patterns in disturbed and rapidly evolving
communities {[}Rominger \emph{et al.} (2015); harte2011{]}. Just like
deviations from UNTB, we cannot know the cause of these departures from
theoretical predictions without adding tests of theories and metrics
that capture temporal dynamics.

\hypertarget{inferring-non-equilibrium-dynamics}{%
\section{Inferring non-equilibrium
dynamics}\label{inferring-non-equilibrium-dynamics}}

Unlocking insight into the dynamics underlying community assembly is key
to overcoming the limitations of analyzing ahistorical patterns with
equilibrial theory. While the fossil record could be used for this task,
it has limited temporal, spatial, and taxonomic resolution. Here we
instead focus on population/phylogenetic insights into rates of change
of populations and species because while there are real limitations in
the accuracy and resolution of temporal dynamics with population genetic
(cite) and phylogenetic (cite) methods, they can, in principle, be
applied to any extant group. Additionally, despite limitations in
resolving detailed temporal dynamics, robust metrics of deviation from
simple, stationary birth-death and/or speciation-extinction processes
have been well-established and widely used for population genetic and
phylogenetic data (cite about Tajima D, gamma stats, etc).

\hypertarget{current-efforts-to-integrate-evolution-into-ecological-theory}{%
\section{Current efforts to integrate evolution into ecological
theory}\label{current-efforts-to-integrate-evolution-into-ecological-theory}}

While quantitatively integrating theory from ecology, population
genetics, and phylogenetics has not yet been achieved, existing efforts
to synthesize perspectives from evolution and ecology point toward
promising directions despite being hindered by one or more general
issues: 1) lack of a solid theoretical foundation, 2) inability to
distinguish multiple competing alternative hypotheses, 3) lack of
comprehensive genetic/genomic data, and 4) lack of bioinformatic
approaches to resolve species and their abundances with high throughput
sequencing data.

Phylogenetic information has been incorporated into studies of the UNTB
to better understand its ultimate equilibrium (Jabot \& Chave 2009;
Burbrink \emph{et al.} 2015). However, phylogenetic reasoning also
points out the flaws in the UNTB's presumed equilibrium (Ricklefs 2006).
Attempts to correct the assumed dynamics of UNTB through ``protracted
speciation'' (Rosindell \emph{et al.} 2010) are promising, and while
their implications for diversification have been considered (Etienne \&
Rosindell 2011), these predictions have not been integrated with
demographic and phylogeographic approaches (e.g., (Edwards \& Beerli
2000; Charlesworth 2010; Prado-Martinez \emph{et al.} 2013)) that have
the potential to validate or falsify presumed mechanisms of lineage
divergence. Such demographic studies, particularly phylogeographic
investigations of past climate change, have highlighted the
non-equilibrium responses of specific groups to perturbations (Hickerson
\& Cunningham 2005; Smith \emph{et al.} 2012), but no attempt has been
made to scale up these observations to implications at the level of
entire communities. The recent growth in joint studies of genetic and
species diversity (Vellend 2005; Papadopoulou \emph{et al.} 2011) have
been useful in linking population genetics with ecological and
biogeographic concepts.

Studies have also used chronosequences or the fossil record in
combination with neutral and/or statistical theory to investigate
changes over geologic time in community assembly mechanisms (Wagner
\emph{et al.} 2006; Rominger \emph{et al.} 2015). While these studies
have documented interesting shifts in assembly mechanisms, including
departures from equilibrium likely resulting from evolutionary
innovations, understanding exactly how the evolution of innovation is
responsible for these departures cannot be achieved without more
concerted integration with insights from evolutionary theory.

\hypertarget{data-and-code-availability}{%
\subsection{Data and Code
Availability}\label{data-and-code-availability}}

All data and code needed to reproduce the results of this manuscript are
available at \url{https://github.com/ajrominger/EcoEvoNonStat} and a
detailed description of the analytical approach is available in the
supplement.

\clearpage

\hypertarget{references}{%
\section*{References}\label{references}}
\addcontentsline{toc}{section}{References}

\hypertarget{refs}{}
\begin{CSLReferences}{1}{0}
\leavevmode\vadjust pre{\hypertarget{ref-baldridge2016}{}}%
Baldridge E, Harris DJ, Xiao X, White EP (2016) An extensive comparison
of species-abundance distribution models. \emph{PeerJ}, \textbf{4},
e2823.

\leavevmode\vadjust pre{\hypertarget{ref-blonder2015}{}}%
Blonder B, Nogués-Bravo D, Borregaard MK, \emph{et al.} (2015) Linking
environmental filtering and disequilibrium to biogeography with a
community climate framework. \emph{Ecology}, \textbf{96}, 972--985.

\leavevmode\vadjust pre{\hypertarget{ref-bohmann2014}{}}%
Bohmann K, Evans A, Gilbert MTP, \emph{et al.} (2014) Environmental DNA
for wildlife biology and biodiversity monitoring. \emph{Trends in
Ecology \& Evolution}, \textbf{29}, 358--367.

\leavevmode\vadjust pre{\hypertarget{ref-burbrink2015}{}}%
Burbrink FT, McKelvy AD, Pyron RA, Myers EA (2015) Predicting community
structure in snakes on eastern nearctic islands using ecological neutral
theory and phylogenetic methods. \emph{Proceedings of the Royal Society
B: Biological Sciences}, \textbf{282}, 20151700.

\leavevmode\vadjust pre{\hypertarget{ref-charlesworth2010}{}}%
Charlesworth D (2010) Don't forget the ancestral polymorphisms.
\emph{Heredity}, \textbf{105}, 509--510.

\leavevmode\vadjust pre{\hypertarget{ref-chesson2000}{}}%
Chesson P (2000) Mechanisms of maintenance of species diversity.
\emph{Annu. Rev. Ecol. Syst.}, \textbf{31}, 343--366.

\leavevmode\vadjust pre{\hypertarget{ref-dodsworth2015}{}}%
Dodsworth S (2015) Genome skimming for next-generation biodiversity
analysis. \emph{Trends in plant science}, \textbf{20}, 525--527.

\leavevmode\vadjust pre{\hypertarget{ref-edwards2000}{}}%
Edwards SV, Beerli P (2000) Perspective: Gene divergence, population
divergence, and the variance in coalescence time in phylogeographic
studies. \emph{Evolution}, \textbf{54}, 1839--1854.

\leavevmode\vadjust pre{\hypertarget{ref-etienne2007}{}}%
Etienne RS (2007) A neutral sampling formula for multiple samples and an
'exact' test of neutrality. \emph{Ecology letters}, \textbf{10},
608--618.

\leavevmode\vadjust pre{\hypertarget{ref-etienne2011}{}}%
Etienne RS, Rosindell J (2011) Prolonging the past counteracts the pull
of the present: Protracted speciation can explain observed slowdowns in
diversification. \emph{Systematic Biology}, syr091.

\leavevmode\vadjust pre{\hypertarget{ref-gibson2014}{}}%
Gibson J, Shokralla S, Porter TM, \emph{et al.} (2014) Simultaneous
assessment of the macrobiome and microbiome in a bulk sample of tropical
arthropods through DNA metasystematics. \emph{Proceedings of the
National Academy of Sciences}, \textbf{111}, 8007--8012.

\leavevmode\vadjust pre{\hypertarget{ref-harte2011}{}}%
Harte J (2011) \emph{The maximum entropy theory of ecology}. Oxford
University Press.

\leavevmode\vadjust pre{\hypertarget{ref-hickerson2005}{}}%
Hickerson MJ, Cunningham CW (2005) Contrasting quaternary histories in
an ecologically divergent pair of low-dispersing intertidal fish
(xiphister) revealed by multi-locus {DNA} analysis. \emph{Evolution},
\textbf{59}, 344--360.

\leavevmode\vadjust pre{\hypertarget{ref-hubbell2001}{}}%
Hubbell SP (2001) \emph{The unified neutral theory of biodiversity and
biogeography}. Princeton University Press.

\leavevmode\vadjust pre{\hypertarget{ref-jabot2009}{}}%
Jabot F, Chave J (2009) Inferring the parameters of the neutral theory
of biodiversity using phylogenetic information and implications for
tropical forests. \emph{Ecol. Lett.}, \textbf{12}, 239--248.

\leavevmode\vadjust pre{\hypertarget{ref-ji2013}{}}%
Ji Y, Ashton L, Pedley SM, \emph{et al.} (2013) Reliable, verifiable and
efficient monitoring of biodiversity via metabarcoding. \emph{Ecology
letters}, \textbf{16}, 1245--1257.

\leavevmode\vadjust pre{\hypertarget{ref-leray2015}{}}%
Leray M, Knowlton N (2015) DNA barcoding and metabarcoding of
standardized samples reveal patterns of marine benthic diversity.
\emph{Proceedings of the National Academy of Sciences}, \textbf{112},
2076--2081.

\leavevmode\vadjust pre{\hypertarget{ref-linard2015}{}}%
Linard B, Crampton-Platt A, Gillett CP, Timmermans MJ, Vogler AP (2015)
Metagenome skimming of insect specimen pools: Potential for comparative
genomics. \emph{Genome biology and evolution}, \textbf{7}, 1474--1489.

\leavevmode\vadjust pre{\hypertarget{ref-liu2016}{}}%
Liu S, Wang X, Xie L, \emph{et al.} (2016) Mitochondrial capture
enriches mito-DNA 100 fold, enabling PCR-free mitogenomics biodiversity
analysis. \emph{Molecular ecology resources}, \textbf{16}, 470--479.

\leavevmode\vadjust pre{\hypertarget{ref-mcgill2003}{}}%
McGill B (2003) Strong and weak tests of macroecological theory.
\emph{Oikos}, \textbf{102}, 679--685.

\leavevmode\vadjust pre{\hypertarget{ref-mcgill2007}{}}%
McGill BJ, Etienne RS, Gray JS, \emph{et al.} (2007) Species abundance
distributions: Moving beyond single prediction theories to integration
within an ecological framework. \emph{Ecol. Lett.}, \textbf{10},
995--1015.

\leavevmode\vadjust pre{\hypertarget{ref-neill2009}{}}%
Neill C, Daufresne T, Jones CG (2009) A competitive coexistence
principle? \emph{Oikos}, \textbf{118}, 1570--1578.

\leavevmode\vadjust pre{\hypertarget{ref-papadopoulou2011}{}}%
Papadopoulou A, Anastasiou I, Spagopoulou F, \emph{et al.} (2011)
Testing the {Species--Genetic} diversity correlation in the aegean
archipelago: Toward a {Haplotype-Based} macroecology? \emph{Am. Nat.},
\textbf{178}, 241--255.

\leavevmode\vadjust pre{\hypertarget{ref-prado-martinez2013}{}}%
Prado-Martinez J, Sudmant PH, Kidd JM, \emph{et al.} (2013) Great ape
genetic diversity and population history. \emph{Nature}, \textbf{499},
471--475.

\leavevmode\vadjust pre{\hypertarget{ref-pueyo2007}{}}%
Pueyo S, He F, Zillio T (2007) The maximum entropy formalism and the
idiosyncratic theory of biodiversity. \emph{Ecology Letters},
\textbf{10}, 1017--1028.

\leavevmode\vadjust pre{\hypertarget{ref-rabosky2009}{}}%
Rabosky DL (2009) Ecological limits and diversification rate:
Alternative paradigms to explain the variation in species richness among
clades and regions. \emph{Ecol. Lett.}, \textbf{12}, 735--743.

\leavevmode\vadjust pre{\hypertarget{ref-ricklefs2006}{}}%
Ricklefs RE (2006) The unified neutral theory of biodiversity: Do the
numbers add up? \emph{Ecology}, \textbf{87}, 1424--1431.

\leavevmode\vadjust pre{\hypertarget{ref-rominger2015}{}}%
Rominger AJ, Goodman KR, Lim JY, \emph{et al.} (2015) Community assembly
on isolated islands: Macroecology meets evolution. \emph{Glob. Ecol.
Biogeogr.}

\leavevmode\vadjust pre{\hypertarget{ref-meteR}{}}%
Rominger AJ, Merow C (2017) meteR: An r package for testing the maximum
entropy theory of ecology. \emph{Methods in Ecology and Evolution},
\textbf{8}, 241--247.

\leavevmode\vadjust pre{\hypertarget{ref-rosindell2010}{}}%
Rosindell J, Cornell SJ, Hubbell SP, Etienne RS (2010) Protracted
speciation revitalizes the neutral theory of biodiversity. \emph{Ecol.
Lett.}, \textbf{13}, 716--727.

\leavevmode\vadjust pre{\hypertarget{ref-sepkoski1984}{}}%
Sepkoski JJ (1984) A kinetic model of phanerozoic taxonomic diversity.
{III}. {Post-Paleozoic} families and mass extinctions.
\emph{Paleobiology}, \textbf{10}, 246--267.

\leavevmode\vadjust pre{\hypertarget{ref-shokralla2015}{}}%
Shokralla S, Porter TM, Gibson JF, \emph{et al.} (2015) Massively
parallel multiplex DNA sequencing for specimen identification using an
illumina MiSeq platform. \emph{Scientific reports}, \textbf{5}, 9687.

\leavevmode\vadjust pre{\hypertarget{ref-smith2012}{}}%
Smith BT, Amei A, Klicka J (2012) Evaluating the role of contracting and
expanding rainforest in initiating cycles of speciation across the
isthmus of panama. \emph{Proc. Biol. Sci.}, \textbf{279}, 3520--3526.

\leavevmode\vadjust pre{\hypertarget{ref-taberlet2012}{}}%
Taberlet P, Coissac E, Pompanon F, Brochmann C, Willerslev E (2012)
Towards next-generation biodiversity assessment using DNA metabarcoding.
\emph{Molecular Ecology}, \textbf{21}, 2045--2050.

\leavevmode\vadjust pre{\hypertarget{ref-tilman2004}{}}%
Tilman D (2004) Niche tradeoffs, neutrality, and community structure: A
stochastic theory of resource competition, invasion, and community
assembly. \emph{Proc. Natl. Acad. Sci. U. S. A.}, \textbf{101},
10854--10861.

\leavevmode\vadjust pre{\hypertarget{ref-vellend2005amnat}{}}%
Vellend M (2005) Species diversity and genetic diversity: Parallel
processes and correlated patterns. \emph{Am. Nat.}, \textbf{166},
199--215.

\leavevmode\vadjust pre{\hypertarget{ref-venkataraman2015}{}}%
Venkataraman A, Bassis CM, Beck JM, \emph{et al.} (2015) Application of
a neutral community model to assess structuring of the human lung
microbiome. \emph{MBio}, \textbf{6}.

\leavevmode\vadjust pre{\hypertarget{ref-wagner2006}{}}%
Wagner PJ, Kosnik MA, Lidgard S (2006) Abundance distributions imply
elevated complexity of post-paleozoic marine ecosystems. \emph{Science},
\textbf{314}, 1289--1292.

\leavevmode\vadjust pre{\hypertarget{ref-wallington2005}{}}%
Wallington TJ, Hobbs RJ, Moore SA (2005) Implications of current
ecological thinking for biodiversity conservation: A review of the
salient issues. \emph{Ecol. Soc.}, \textbf{10}.

\leavevmode\vadjust pre{\hypertarget{ref-white2012}{}}%
White EP, Thibault KM, Xiao X (2012) Characterizing species abundance
distributions across taxa and ecosystems using a simple maximum entropy
model. \emph{Ecology}, \textbf{93}, 1772--1778.

\leavevmode\vadjust pre{\hypertarget{ref-zhou2013}{}}%
Zhou X, Li Y, Liu S, \emph{et al.} (2013) Ultra-deep sequencing enables
high-fidelity recovery of biodiversity for bulk arthropod samples
without PCR amplification. \emph{Gigascience}, \textbf{2}, 4.

\end{CSLReferences}

\end{document}
